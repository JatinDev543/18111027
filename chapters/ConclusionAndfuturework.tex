\chapter{Conclusions and Future Work}\label{Chapter6:Conclusion and Future Work}
There have been various enhancements and innovations in the prefetching technology since its introduction. Our research has dealt with SPP and applied critical load address-based enhancements to it. We have conducted a thorough evaluation using different configurations to understand the overall behaviour of the prefetcher. We show that with the critical addresses alone, we can get better performance than SPP by properly adjusting the parameters. Finding the optimal set of parameters is a big challenge and we have set these empirically.
 
Our proposal achieves performance speedup in the single-core configuration with more aggressive prefetching algorithms. In the multi-core configurations, aggressive prefetching results in performance degradation due to pollution and congestion. Our proposal is able to maintain approximately same performance as SPP with the less bandwidth proposal. In most of our evaluation of the less bandwidth proposal,  we achieve performance approximately same as SPP with less bandwidth consumption. Even in the configuration where we pass only the critical load addresses to SPP while using its own lookahead mechanism, we are able to maintain the performance of SPP with lower bandwidth consumption.
 
We observe that managing depth along with pollution holds the key to getting better performance improvement. Our critical load address enhancement to SPP works as an in-built prefetch filter because it generates prefetches only for the critical IPs.
The problem of finding the optimal set of parameters or using dynamic methods for this purpose is an open question. We feel that there is a lot of scope for improvement and there are various areas where we can conduct further studies. Some of these are listed in the following.
 \begin{itemize}
  \item  We have only done experiments with SPP and seen the effects of critical IPs on SPP only. This can be extended to other prefetchers.
  \item We have observed that increasing the depth of prefetching increases performance along with increasing cache pollution. To get better results, we need to have some structure to compute and reduce cache pollution by approximating cache pollution and using it as a feedback while prefetching~\cite{Feedback Directed Prefetching}. Also, advanced prefetch filters can be used to identify and discard the less important prefetches~\cite{Perceptron filtering}.
  \item Although we are able to limit the performance degradation of multi-programmed workloads using our low bandwidth proposal, further work can be done to have a dynamic combination of high and low bandwidth proposals for achieving better or at least equal performance compared to SPP while optimizing bandwidth consumption.
  \item As already discussed, more research needs to be conducted for automatically converging to the optimal parameter set at run-time. Some alternate parameters for efficient computation of lookahead can also be thought about.
\end{itemize}